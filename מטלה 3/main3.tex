\documentclass{article}
\usepackage[utf8x]{inputenc}
\usepackage[english,hebrew]{babel}
\usepackage{amsmath} 
\selectlanguage{hebrew}
\usepackage[top=2cm,bottom=2cm,left=2.5cm,right=2cm]{geometry}
\usepackage{graphicx}
\graphicspath{ {./images/} }
\usepackage{wrapfig}
\usepackage{empheq}
\usepackage{dashbox}

\newcommand{\image}[2]{
    \begin{align*}
        \centering
        \includegraphics[scale=#2]{#1}
    \end{align*}
}

\newcommand{\qimage}[2]{
    \hrule
    \image{#1}{#2}
    \hrule
    \vspace{8pt}
}

\title{נבחרת ישראל הצעירה בפיזיקה - מטלה 3}
\author{איתי מור}
\begin{document}
\maketitle

\newcommand{\derivative}[2]{\frac{d\,#1}{d\,#2}}

\section*{קינמטיקה מתקדמת}
\subsection*{נגזרת של וקטור}
\begin{enumerate}
    \item נגזרת של וקטור היא וקטור.\\
    ניתן לראות זאת מהנוסחה:
    \begin{flalign*}
        &\derivative{\vec{r}}{t} = \lim_{dt \to 0} \frac{\vec{r}(t+dt)-\vec{r}(t)}{dt}&&
    \end{flalign*}
    ההפרש בין 2 וקטורים )המונה( הוא וקטור, וחלוקה של וקטור בסקלר זה גם וקטור.
    % \image{images/vector_derivative_diagram.png}{0.4}

    \item 
    \begin{flalign*}
        &\derivative{\left[ x(t)\hat{x} + y(t)\hat{y} + z(t)\hat{z}\right] }{t}
            = \lim_{dt \to 0} \frac{x(t+dt)\hat{x} - x(t)\hat{x} + y(t+dt)\hat{y} - y(t)\hat{y} + z(t+dt)\hat{z} - z(t)\hat{z}}{dt}&&\\
        &=\hat{x}\lim_{dt \to 0}{\frac{x(t+dt)-x(t)}{dt}} + \hat{y}\lim_{dt\to 0}{\frac{y(t+dt)-y(t)}{dt}} + \hat{z}\lim_{dt\to0}{\frac{z(t+dt)-z(t)}{dt}}&&\\
        &=\hat{x}\cdot\derivative{x(t)}{t} + \hat{y}\cdot\derivative{y(t)}{t} + \hat{z}\cdot\derivative{z(t)}{t}&&\\
        &=\left( {\dot{x}(t)}, {\dot{y}(t)}, {\dot{z}(t)} \right)
    \end{flalign*}
    כלומר רכיבי הנגזרת של וקטור, זה נגזרות רכיבי הוקטור, או במילים אחרות:
    ניתן לגזור וקטור לפי כל רכיב בנפרד.
\end{enumerate}

\subsection*{תנועה מעגלית}
\begin{enumerate}
    \item אחרי יחידת זמן אחת, הזווית של הגוף גדלה ב-$\omega$.
    ולכן אחרי זמן
    $t$
    הזווית גדלה ב-$\omega t$.
    \item 
    הזווית של וקטור המיקום היא
    $\omega t$
    כי זו הזוית שהגוף עובר ב-$t$ שניות.
    לכן הוקטור שווה ל:
    \begin{flalign*}
        & \vec{r}(t) = \langle R\, \angle\, \omega t\rangle = (R\cos(\omega t), R\sin(\omega t))&&
    \end{flalign*}
\image{circular_movement_diagram1.png}{0.4}
    \item מכיוון שמהירות היא השינוי בזמן של המיקום, כדי לקבל את וקטור המהירות, אפשר לגזור את וקטור המיקום.\\
    בשאלה הקודמת ראינו שאפשר לגזור כל רכיב בנפרד ולכן:
    \begin{flalign*}
        & \vec{v}(t) = \derivative{\vec{r}(t)}{t} = ((R\cos(\omega t))', (R\sin(\omega t))') = 
        (-R\omega\sin(\omega t), R\omega\cos(\omega t)) &&
    \end{flalign*}
    ניתן לראות שוקטור המהירות מאונך לוקטור המיקום מכיוון שהמכפלה הסקלרית שלהם שווה ל-0:
    \begin{flalign*}
        & \vec{r}(t) \cdot \vec{v}(t) = -R^2\omega \sin(\omega t)\cos(\omega t) + R^2\omega\sin(\omega t)\cos(\omega t) = 0&&
    \end{flalign*}

    \item 
    כדי לקבל את וקטור התאוצה, נגזור את וקטור המהירות:
    \begin{flalign*}
        & \vec{a}(t) = \derivative{\vec{v}(t)}{t} = ((-R\omega\sin(\omega t))', (R\omega\cos(\omega t))') = (-R\omega^2\cos(\omega t), -R\omega^2\sin(\omega t)) = -\omega^2\vec{r}(t)&&
    \end{flalign*}
    כלומר הוקטור הזה הפוך בכיוונו מוקטור המיקום או במילים אחרות, הגוף מאיץ לכיוון מרכז מעגל התנועה.

    \item
    \image{circular_movement_diagram2.png}{0.4}
    
    \item
    \textbf{\underline{ביטוי לפי מהירות זוויתית:}}\\
    בסעיף 4 ראינו שניתן להביע את וקטור התאוצה הצנטרפיטלית ברגע מסוים ע"י וקטור המיקום והמהירות הזוויתית כ $-\omega^2\vec{r}(t)$
    ולכן גודל התאוצה הצנטרפיטלית הוא:
    \begin{flalign*}
        & \left|\vec{a}(t)\right| = \lvert-\omega^2\vec{r}(t)\rvert = \omega^2\lvert\vec{r}(t)\rvert = R\omega^2&&
    \end{flalign*}
    מבחינת יחידות זה מסתדר מכיוון שזווית זה גודל חסר יחידות )זה מבטא יחס בין 2 אורכים( ולכן:
    \begin{flalign*}
        &\left[R\omega^2\right] = [m]\cdot \left[ \frac{1^2}{s^2}\right] = \left[ \frac{m}{\displaystyle s^2} \right] &&
    \end{flalign*}
    ואכן אלו יחידות של תאוצה.\\\\
    \textbf{\underline{ביטוי לפי מהירות משיקית:}}\\
    נחשב את גודל המהירות המשיקית:
    \begin{flalign*}
        & \left|\vec{v}(t)\right| =\sqrt{(-R\omega\sin(\omega t))^2 + (R\omega\cos(\omega t))^2} =
         \sqrt{(R\omega)^2\left(\sin^2(\omega t) +\cos^2(\omega t)\right)} = \sqrt{(R\omega)^2\cdot 1} = R\omega &&
    \end{flalign*}
    לכן גודל התאוצה הצנטריפטלית $R\omega^2$ שווה גם ל $\frac{\displaystyle v^2}{R}$
    כאשר $v$ הוא גודל המהירות המשיקית.

    \item
    נסמן ב-$a_c$ את התאוצה הצנריפטלית.\\
    מכיוון שמתיחות החוט זה הכח היחיד שפועל על הגוף, זהו גם בהכרח הכח הצנטריפטלי ששוה ל-$ma_c$
    כלומר:
    \begin{flalign*}
        & T = mR\omega^2 = \frac{\displaystyle mv^2}{R}
    \end{flalign*}
\end{enumerate}

\end{document}